%%%%%%%%%%%%%%%%%%%%%%%%%%%%%%%%%%%%%%%%%
% Cheatsheet
% LaTeX Template
% Version 1.0 (12/12/15)
%
% This template has been downloaded from:
% http://www.LaTeXTemplates.com
%
% Original author:
% Michael Müller (https://github.com/cmichi/latex-template-collection) with
% extensive modifications by Vel (vel@LaTeXTemplates.com)
%
% License:
% The MIT License (see included LICENSE file)
%
%%%%%%%%%%%%%%%%%%%%%%%%%%%%%%%%%%%%%%%%%

%----------------------------------------------------------------------------------------
%	PACKAGES AND OTHER DOCUMENT CONFIGURATIONS
%----------------------------------------------------------------------------------------

\documentclass[11pt]{scrartcl} % 11pt font size

\usepackage[utf8]{inputenc} % Required for inputting international characters
\usepackage[T1]{fontenc} % Output font encoding for international characters

\usepackage[margin=0pt, landscape]{geometry} % Page margins and orientation

\usepackage{graphicx} % Required for including images

\usepackage{color} % Required for color customization
\definecolor{mygray}{gray}{.75} % Custom color

\usepackage{url} % Required for the \url command to easily display URLs

\usepackage[ % This block contains information used to annotate the PDF
colorlinks=false, 
pdftitle={Slice Cheatsheet}, 
pdfauthor={Nervos Network}, 
pdfsubject={Cheatsheet for slice syntax in different languages}, 
pdfkeywords={Cheatsheet, Programming, Slice}
]{hyperref}

\setlength{\unitlength}{1mm} % Set the length that numerical units are measured in
\setlength{\parindent}{0pt} % Stop paragraph indentation

\renewcommand{\dots}{\ \dotfill{}\ } % Fills in the right amount of dots

\newcommand{\command}[2]{#1~\dotfill{}~#2\\} % Custom command for adding a shorcut

\newcommand{\sectiontitle}[1]{\paragraph{#1} \ \\} % Custom command for subsection titles

%----------------------------------------------------------------------------------------

\begin{document}

\begin{picture}(297,210) % Create a container for the page content

%----------------------------------------------------------------------------------------
%	TITLE SECTION 
%----------------------------------------------------------------------------------------

\put(10,200){ % Position on the page to put the title
\begin{minipage}[t]{210mm} % The size and alignment of the title
\section*{Slice Cheatsheet -- Syntaxes in Different Languages} % Title
\end{minipage}
}

%----------------------------------------------------------------------------------------
%	FIRST COLUMN SPECIFICATION
%----------------------------------------------------------------------------------------

\put(10,180){ % Divide the page
\begin{minipage}[t]{85mm} % Create a box to house text

%----------------------------------------------------------------------------------------
%	HEADING
%----------------------------------------------------------------------------------------

\sectiontitle{Definitions}
			
arr = [1, 2, 3, 4, 5, 6, 7, 8]\\
			
\sectiontitle{Rust}
			
\command{arr[1..5]}{[2, 3, 4, 5]}
\command{arr[1..=5]}{[2, 3, 4, 5, 6]}

\sectiontitle{Golang}

\command{arr[1:5]}{[2, 3, 4, 5]}

\sectiontitle{Python}
\command{arr[1:5]}{[2, 3, 4, 5]}
\command{arr[1:5:2]}{[2, 4]}
\command{arr[1:5:3]}{[2, 5]}
\command{arr[2..-1]}{[3, 4, 5, 6, 7]}
\command{arr[2..-2]}{[3, 4, 5, 6]}
\command{arr[-5..-1]}{[4, 5, 6, 7]}
\command{arr[5:1:-1]}{[6, 5, 4, 3]}
\command{arr[::-1]}{[8, 7, 6, 5, 4, 3, 2, 1]}

%----------------------------------------------------------------------------------------

\end{minipage} % End the first column of text
} % End the first division of the page

%----------------------------------------------------------------------------------------
%	SECOND COLUMN SPECIFICATION 
%----------------------------------------------------------------------------------------

\put(105,180){ % Divide the page
\begin{minipage}[t]{85mm} % Create a box to house text

%----------------------------------------------------------------------------------------
%	HEADING
%----------------------------------------------------------------------------------------

\sectiontitle{Ruby}

\command{arr[1, 5]}{[2, 3, 4, 5, 6]}
\command{arr[2, 5]}{[3, 4, 5, 6, 7]}
\command{arr[1..5]}{[2, 3, 4, 5, 6]}
\command{arr[2..5]}{[3, 4, 5, 6]}
\command{arr[2...5]}{[3, 4, 5]}
\command{arr[2..-1]}{[3, 4, 5, 6, 7, 8]}
\command{arr[2..-2]}{[3, 4, 5, 6, 7]}
\command{arr[-5..-1]}{[4, 5, 6, 7, 8]}
\command{arr[2...-1]}{[3, 4, 5, 6, 7]}
					
\sectiontitle{ECMAScript TC39}

\command{arr[1:4]}{[2, 3, 4]}
\command{arr[1:4:2]}{[2, 4]}

const obj = \{ 0: 'a', 1: 'b', 2: 'c', 3: 'd', length: 4 \}; \\

\command{obj[1:3]}{['b', 'c']}

%----------------------------------------------------------------------------------------

\end{minipage} % End the second column of text
} % End the second division of the page

%----------------------------------------------------------------------------------------
%	THIRD COLUMN SPECIFICATION 
%----------------------------------------------------------------------------------------

\put(200,180){ % Divide the page
\begin{minipage}[t]{85mm} % Create a box to house tex

%----------------------------------------------------------------------------------------
%	FOOTNOTE
%----------------------------------------------------------------------------------------

\vspace{\baselineskip}
\linethickness{0.5mm} % Thickness of the footer line
{\color{mygray}\line(1,0){30}} % Print the line with a custom color

\footnotesize{
Maintained by the Nervos team, 2020\\ 
\url{https://www.nervos.org/}\\
				
Released under the MIT license.
}

%----------------------------------------------------------------------------------------

\end{minipage} % End the third column of text
} % End the third division of the page
\end{picture} % End the container for the entire page

%----------------------------------------------------------------------------------------

\end{document}
